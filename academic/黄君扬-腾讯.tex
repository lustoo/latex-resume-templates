\documentclass{sorahjy_cv}
%\usepackage[colorlinks,linkcolor=red]{hyperref}
\usepackage{enumitem}
\usepackage{graphicx}
\usepackage{xeCJK}
\usepackage{xltxtra}
\XeTeXlinebreaklocale "zh"


\begin{document}
\pagestyle{empty}


% Name, phone, email, etc.
\begin{cvHeader}
	\printName{黄 \ 君 \ 扬 \ \ {\large \textbf{意向}:软件开发-后台开发方向}}
	\printPhone{(+86)180-1900-2473 \qquad \qquad \qquad}
	\printEmail{sorahjy@gmail.com \qquad \qquad \qquad \ }
	\printWebsite{https://blog.sorahjy.com}
	\printGithub{https://github.com/sorahjy}
\end{cvHeader}


%
%Strengths
%

\sectionTitle{Strengths}
\begin{description}{}
	\item{\ \ }\textbf{有进取心, 快速学习, 热爱编程, 脚踏实地, 擅于合作, 良好的团队精神, 批判性思维, 乐于交际。}
\end{description}


%
% Education
%

\sectionTitle{Education}
\begin{sectionContentSimple}{上海理工大学}{Sep. 2015 - Present}
	\item 本科大三在读 \quad 专业:计算机科学与技术 \quad GPA: 4.03 \quad 排名: \textbf{1/107} 
\end{sectionContentSimple}

%
% Experience
%

% \sectionTitle{Experience}

% \begin{sectionContentNormal}{Hotel Management System.}{Oct. 2017 - Jan. 2018}{Member of Academic Dept.}
% 	\item XXXXXXXXXXXXXXXXXXXXXXXXX
% \end{sectionContentNormal}

% \begin{sectionContentNormal}{Collaborative Filtering Recommendation Model based on Convolutional Denoising Auto Encoder.}{Oct. 2017 - Present }{Intern}
% 	\item Did research on XXXXXXXXXXXXXX
% \end{sectionContentNormal}

% \begin{sectionContentNormal}{A Portable Integrated Identity Authentication Method.}{Dec. 2017 - Mar. 2018}{Student Research Assistant}
% 	\item Did research project about XXXXXXXXXXXXX
% \end{sectionContentNormal}


%
% Skills
%

\sectionTitle{Skills}
\begin{description}{}
	\item{\ \ 熟悉的语言: } \textbf{Java} \quad 了解的语言: Python / C++ / C / Kotlin \quad \textbf{英语: CET-4 590 | CET-6 534}
	\item{\ \ }熟悉算法与数据结构,多次参加算法竞赛并获奖,具有良好的算法素养。
	\item{\ \ }了解传统机器学习算法的实现方法,并多次在数学建模竞赛中应用。
	\item{\ \ }良好的数据库原理基础,熟悉数据库的并发控制和数据库的恢复。了解SQL语句,有Mysql的使用经验。
	\item{\ \ }良好的计算机网络基础,熟悉UDP和TCP协议,了解常见的网络I/O模型。
	\item{\ \ }良好的操作系统基础,熟悉操作系统的调度算法、线程/进程的区别。
	\item{\ \ }对设计模式有所了解。
\end{description}

%
% Awards
%

\sectionTitle{Grants \& Awards}
\begin{description}{}
	\item{\ \ }\textbf{连续四次}校学习优秀奖学金\textbf{一等奖}\quad \quad 2017年\textbf{上海市}奖学金
	\item{\textbf{\ \ 算法:}} 2017年第42届ACM-ICPC亚洲区域赛(青岛) \hfill \textbf{银奖}
	\item{\ \ \quad \quad \quad } 2017年第八届蓝桥杯Java软件开发 \hfill \textbf{省赛(上海)一等奖 \& 全国 二等奖}
	\item{\ \ \quad \quad \quad } 2018年第九届蓝桥杯Java软件开发 \hfill \textbf{省赛(上海)一等奖}
	\item{\ \ \quad \quad \quad } 2018年第三届中国高校计算机团体程序设计天梯赛 \hfill \textbf{全国三等奖}
	\item{\textbf{\ \ 建模:}} 2017年全国大学生数学建模竞赛上海赛区本科组 \hfill \textbf{三等奖}
	\item{\ \ \quad \quad \quad } 2017年APMCM亚太地区大学生数学建模竞赛 \hfill \textbf{二等奖}
	\item{\ \ \quad \quad \quad } 2018年美国大学生数学建模竞赛 \hfill \textbf{二等奖}
\end{description}


%
% Projects
%


\sectionTitle{Projects \& Experience}
\begin{sectionContentSimple}{一种便携的集成身份认证方法}{Dec. 2017 - Apr. 2018}
	\item \url{https://github.com/sorahjy/Identity-Authentication-WeAPP}
	\item \textbf{个人项目,独立开发。} 提供了一种多层次、多维度且易定制、易推广的身份认证服务。
	\item \textbf{核心技术}:基于时间的一次性密码(TOTP)、人脸识别、全球定位、微信用户唯一识别码OpenID。
	\item \textbf{项目成果}:目前以微信小程序形式上线,可以通过扫描二维码使用。
\end{sectionContentSimple}

\begin{sectionContentSimple}{NTM文档协同编辑系统}{Mar. 2018 - \ \ Present \ }
	\item \url{https://github.com/sorahjy/Collaboration}
	\item \textbf{项目组组长,团队开发。}软件协同设计课程大作业,学习并了解工程项目的流程。目前正在开发中。
	\item \textbf{创新点}:设计并提出了“排他编辑锁”机制,来维护多人编辑同一处文本时的数据一致性。
	\item \textbf{目前成果}:了解软件协同开发流程,协调并参与组员之间的沟通,认识到工程项目中文档的重要性。
\end{sectionContentSimple}


% \begin{sectionContentNormal}{融合卷积降噪自动编码器的协同过滤推荐系统}{Apr. 2017 - \ \ Present \ }{霍教授实验室, 本研究得到国家自然科学基金项目(61003031)的资助。 }
% 	\item \textbf{研究助理。} 配置实验室的实验环境, 准备数据集并对实验进行分析. 校验和编辑论文.
% \end{sectionContentNormal}

%
% Publications
%

%\sectionTitle{Publications}
%\begin{sectionContentNaive}
%\item XXXX, XXXX, XXXX, \underline{Your Name}, XXXXX, ... "XXXXXXX XXXXXXX XXXXXXX XXXX" In XXX, pp. 15-24. 2017.
%\end{sectionContentNaive}


%
%Certificates
%

% \sectionTitle{Certificates}
% \begin{description}{}
% 	\item{\ \ } 英语: CET-4 590 | CET-6 534
% 	\item{\ \ } 上海市计算机等级考试二级 C 语言 (优秀,99/100 分)
% \end{description}



%\pagebreak


%\sectionTitle{My photo}
%\centering \includegraphics[height=4in]{photo.jpg}

\end{document}

